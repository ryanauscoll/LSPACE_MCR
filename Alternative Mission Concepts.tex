\section*{1.8 Alternative Mission Concepts}
\addcontentsline{toc}{section}{1.8 Alternative Mission Concepts}

In developing a mission to address Decadal Survey science goal Q10.3b—understanding the long-term controls on the presence of liquid water on terrestrial planets—the team evaluated multiple concepts before selecting a rover mission to Moreux Crater. Concepts considered included: 
\begin{enumerate}
    \item A stationary lander with deep drilling capabilities
    \item A low-orbit radar sounding satellite (orbiter)
    \item A network of surface sensors
    \item Missions to analog bodies such as the Moon or Martian moons
\end{enumerate}

A stationary lander with a deep drill was attractive for directly sampling ice deposits. It could access stratigraphy unavailable to remote instruments, using drills modeled after \textit{Icebreaker} or \textit{Mars Life Explorer}. However, this option was discarded due to high complexity, risk, and mass. Deep drilling is technically challenging—as demonstrated by \textit{InSight}’s failed mole—and a stationary platform lacks mobility. If the lander misses ice-rich terrain by even a few meters, the science return could be negligible. Furthermore, the mission’s objective to understand ice distribution demands mobility to access multiple geological contexts.

An orbiter concept, inspired by \textit{Mars Ice Mapper} and \textit{SHARAD}, would use radar and neutron instruments to detect subsurface ice across broad regions. Its global mapping capability suits long-term water tracking, but orbital instruments lack the resolution to confirm ice at the meter or centimeter scale. Additionally, orbiters infer rather than directly verify ice composition or state. The proposed mission must identify ice usable for ISRU, which requires \textit{in-situ} confirmation of purity, depth, and cover material—measurements best made at the surface.

A distributed sensor network (e.g., 3–5 landers or probes) was also discussed. This architecture provides spatial diversity and redundancy, enabling measurements across different terrains. However, each lander’s payload would be constrained in size and power, limiting the number and quality of instruments. The complexity and cost of multiple Entry, Descent, and Landing (EDL) events also exceeded the team’s Discovery-class constraints. A rover, by contrast, can simulate this network by traversing between sites with a full suite of instruments.

Missions to analog targets, such as the lunar poles or Phobos/Deimos, were ruled out due to poor alignment with the science goal. Lunar water is preserved under very different conditions, and Martian moons cannot measure Mars’s subsurface directly. These targets may yield insights into water delivery or loss, but cannot probe the Martian cryosphere \textit{in situ}.

Ultimately, the rover architecture provided the best balance. It enables surface access, mobility, instrument integration, and adaptability. A single platform can carry GPR, neutron spectroscopy, VNIR, and thermal sensors—all critical to mapping and characterizing water ice. The rover can traverse diverse terrains, collecting high-resolution data at multiple sites to understand both present-day ice stability and the geological controls on its preservation. It also reduces technical risk by leveraging Mars flight heritage.

\textbf{In summary}, alternatives were rejected due to limitations in resolution, cost, risk, or science traceability. The rover mission at Moreux Crater was selected for its ability to meet all measurement requirements efficiently, maximize science return, and support future crewed exploration.