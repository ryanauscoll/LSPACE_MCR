\subsection*{1.9.2 Schedule Basis of Estimate}
\addcontentsline{toc}{subsection}{1.9.3 Budget Basis of Estimate}

Following the completion of the Preliminary Design Review (PDR), the team will progress into creating a more detailed design, including additional subsystems and operational systems, collectively referred to as Phase C. Based on the timeline established for NASA’s \textit{Lucy Mission}, which had a similar budget and duration for Pre-Phase A, Phase A, and Phase B tasks—assumed to have already been completed—the team should be allotted one year to complete Phase C. Therefore, the expected date of completion and approval for the Critical Design Review (CDR) is August 19, 2026. Key Decision Point (KDP) meetings are assumed to occur over a five-day period, shortly before the end of Phases C, D, and E. These meetings will be held at NASA Headquarters in Washington, D.C.

Upon CDR approval, the project will enter Phase D, which includes system assembly, manufacturing, integration, testing, and launch preparation. Using the \textit{Lucy Mission} as a reference, Phase D will begin by June 2027. The team will utilize the preceding year for preparations, including materials acquisition, team coordination, and subsystem prototyping. The rover system must be fully integrated and ready by October 1, 2029, aligning with constraints outlined in the Mission Task Document provided by NASA L’SPACE.

A two-month buffer will be allocated for risk mitigation and final verifications. Launch readiness is set for December 1, 2029. In preparation, the team will travel to Orlando, Florida for a five-day window surrounding launch—arriving two days before and departing two days after.

Following launch, the mission enters Phase E. Based on the transit duration of the \textit{Opportunity Rover}, a seven-and-a-half-month cruise phase is expected before Mars arrival. Upon successful landing, the rover will begin surface operations and data collection over the course of 15 Earth years. The data collected will be transmitted back to Earth and analyzed through 2044. These findings are expected to significantly advance our understanding of Martian hydrology and geology, ultimately supporting future human exploration initiatives.