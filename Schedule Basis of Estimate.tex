\subsection*{1.9.2 Schedule Basis of Estimate}
\addcontentsline{toc}{subsection}{1.9.2 Schedule Basis of Estimate}

Following completion of the Preliminary Design Review (PDR), the team will move into Phase C—focused on developing detailed designs across all subsystems and mission operations. Drawing from the schedule of NASA’s \textit{Lucy Mission}, which had a comparable timeline and scope during its early development, one year is allocated for Phase C completion\footnote{See \cite{lucy_timeline}}. This places the expected Critical Design Review (CDR) milestone on August 19, 2026. Key Decision Point (KDP) meetings are planned to occur within a five-day window near the end of Phases C, D, and E, held at NASA Headquarters in Washington, D.C.\\

With CDR approval, the project will transition into Phase D, which covers system integration, manufacturing, environmental testing, and launch preparation. Based on timelines used for both the Lucy and Mars Exploration Rover programs\footnote{See \cite{lucy_cost, mer_cost}}, Phase D is projected to begin by June 2027. The prior year will be dedicated to material procurement, prototype development, and team-wide coordination in preparation for assembly and qualification testing. The full system must be completed and launch-ready by October 1, 2029, as outlined in the Mission Task Document from NASA L'SPACE.\\

A two-month buffer will be included prior to launch for final systems verification and contingency planning. The scheduled launch readiness date is December 1, 2029. To support this milestone, the team will deploy to Orlando, Florida, for a five-day launch window—arriving two days prior and departing two days after.\\

After launch, the mission enters Phase E. Drawing again from the \textit{Opportunity Rover}'s transit profile, a cruise phase of approximately seven and a half months is anticipated before arrival at Mars. Once surface operations begin, the rover will execute its science campaign for up to 15 Earth years. Throughout this period, data collected on subsurface ice, geology, and environmental conditions will be relayed to Earth and analyzed through at least 2044. These insights will directly contribute to NASA’s long-term human exploration goals by informing resource utilization and risk mitigation strategies for future crewed missions.