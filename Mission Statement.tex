\section*{1.1 Mission Statement}
\addcontentsline{toc}{section}{1.1 Mission Statement}

The goal of this mission is to detect and characterize near-surface water ice within a high-priority Exploration Zone (EZ) on Mars. This effort aims to enable both breakthrough scientific discoveries and to support NASA's long-term objective of sustained human presence on the Red Planet. Specifically, the mission will gather geophysical and compositional data related to shallow subsurface water ice (0–1 meter depth) and associated hydrated minerals. These data are crucial for evaluating in situ resource availability, assessing potential habitability, and reducing risk for future crewed missions.

To accomplish this, the mission will deploy a lightweight, solar-powered robotic rover equipped with a suite of instruments capable of in situ analysis. The rover will traverse a 10 km × 10 km region, targeting diverse terrain types such as impact craters, polygonal ground, and sloped regolith to maximize the probability of detecting preserved ice. Key payload instruments include ground-penetrating radar to detect dielectric contrasts linked to ice layers, neutron spectroscopy for hydrogen abundance mapping, and visible/infrared spectrometers for mineralogical context.

In addition to scientific instrumentation, the rover will carry a compact, human-exploration-relevant astrobiology experiment. This experiment, constrained by mass and volume limitations, will test life detection protocols or contamination mitigation strategies directly applicable to future crewed exploration.

The collected data will enhance our understanding of the Martian cryosphere, inform models of planetary climate evolution, and directly support site selection for In-Situ Resource Utilization (ISRU). This mission will serve as a key stepping stone toward NASA’s Moon to Mars exploration architecture by providing actionable data for landing site assessment, resource extraction, and long-duration surface operations.