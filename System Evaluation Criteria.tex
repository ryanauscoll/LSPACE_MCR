\section*{1.6 System Evaluation Criteria}
\addcontentsline{toc}{section}{1.6 System Evaluation Criteria}

In order to successfully evaluate the design options for the subsystems, criteria based on the mission objectives and environmental effects need to be developed. This criteria will then be used to complete trade studies and finalize a rover design. The criteria for the trade study are heavily influenced by the requirements set previously by the customer and mission goals. \footnote{See \cite{lspace2025mission}} To begin, a general set of criteria has been established to evaluate the general mission requirements, such as mass, dimension, and budget, as well as other principal and important criteria, such as reliability and TRL Level. The list is provided below.\\

\textbf{General Criteria Used for all subsystems:}
\begin{itemize}
    \item Performance 
    \item Cost 
    \item Complexity 
    \item TRL Level 
    \item Manufacturability 
    \item Mass 
    \item Reliability 
    \item Dimensions 
    \item Risk
    \item Power Consumption 
    \item Instrument Usability 
    \item Stability 

\end{itemize}
Manufacturability, Cost, and Complexity are the first three parameters the design will be evaluated by. These parameters weigh heavily on the schedule for a project, one of the primary constraints given by the customer. Therefore, to accomplish the project by the target date, it is important that our final design meets these criteria while maintaining cost within budget.\\

The mass, dimensions, and stability are the main parameters that will constrain our structural design. For our rover to be launchable from Earth it is important that it is able to fit in its stowed position inside the payload volume of the rocket. Due to these constraints, our design must meet its mass and dimensions requirements for the mission. Our design will then prioritize being as small and lightweight as possible without compromising functionality. Stability will be evaluated by the rovers' ability to survive the mechanical loads of the Martian environment.\\

Performance, TRL Level, and Reliability will determine the rovers ability to complete the mission. It is important for the completion of the project that the rover complies with the required TRL Level by the end of its design to meet schedule constraints. Reliability and performance will be evaluated by the ability of the design to complete its science objective and mission life.\\

Power Consumption and Instrument Usability are primarily driven by the power generation and payload subsystems. The power consumption criteria will be graded based on its ability for the rover to be sustained by battery power during the sleep period and the ability to perform scientific operations with the power generated by the solar panels. The instrument usability is based on the ability of the payload to complete the science objectives.\\

Risk will be a parameter that will encompass the mission design as a whole. This risk parameter encompasses risk based on mission schedule, TRL level, reliability and stability. Therefore, this parameter will quantify the risk involved throughout the mission lifecycle as a whole.\\

To evaluate these parameters, there will be a grading system from zero to ten, where ten represents the highest performance and zero indicating failure. This numerical evaluation provides a quantitative basis for comparison; essential for an effective trade study. Additionally, each category will be weighted,allowing for adjustments based on mission needs. The primary goal is to find the most effective and efficient components for the specific mission goals and requirements set. By applying consistent and relevant criteria, the trade study ensures well informed, objective decision making allowing the team to select the most mission-appropriate design option.\\

 