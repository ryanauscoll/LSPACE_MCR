\subsection*{1.9.3 Budget Basis of Estimate}
\addcontentsline{toc}{subsection}{1.9.3 Budget Basis of Estimate}

This mission spans approximately five years and allocates a total of \$420 million\footnote{See \cite{phoenix_cost}}, covering two years of development and three years of operations. The NASA New Index presents the budget estimate in FY2024 dollars and incorporates a 2.5\% annual inflation rate. It is assumed that all critical technologies will reach Technology Readiness Level 6 by the time of the Preliminary Design Review, ensuring that major systems such as propulsion, power, and science instruments are validated in relevant environments and ready for flight hardware integration. This assumption highlights a safe and achievable development timeline. As stated in the mission task document, Pre-phase A and B costs are treated as sunk costs and therefore excluded from this estimate.\\

The total \$420 million lifecycle cost is distributed with approximately \$273 million allocated to development and \$147 million to operations. Within development costs, \$125 million is budgeted for spacecraft systems, \$75 million for payload instruments covering two flight units and one engineering model\footnote{See \cite{mer_cost}}, and \$25 million for a fixed-price launch via NASA’s 2029 planetary rideshare program. A \$27.3 million reserve is applied to development costs for unallocated future expenses, following guidance from the NASA Cost Estimating Handbook. Operations costs include \$80 million for mission support, \$35 million for ground systems, and \$32 million for science team activities. A \$7.4 million reserve is included for operational uncertainties, addressing potential technical or schedule delays at an estimated impact of \$1.8 million per month.\\

The estimate is supported by multiple validation methods. Parametric cost modeling was used to verify the reasonableness of science instrument costs based on analogous missions. It is assumed that labor rates follow NASA's standard burdened salary structures as \$80,000 per year for scientists and engineers, \$60,000 per year for technicians and administrative staff, and \$120,000 per year for management personnel. These rates include salary, fringe benefits, and overhead. Personnel costs also account for necessary travel throughout the mission lifecycle. A total of \$2.1 million is allocated for mission-critical travel, including reviews (such as CDR), launch campaigns, and scientific collaboration meetings. This figure includes a 10\% contingency to accommodate scheduling changes or additional travel requirements\footnote{See \cite{phoenix_press_kit}}.\\

This Budget Basis of Estimate was prepared based on historical data from comparable NASA missions, current economic conditions, and cost modeling tools available at this early phase of mission planning. Assumptions and figures will be refined as the mission progresses into later design phases.