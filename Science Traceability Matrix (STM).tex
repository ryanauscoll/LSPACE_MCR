\section*{1.2 Science Traceability Matrix (STM)}
\addcontentsline{toc}{section}{1.2 Science Traceability Matrix (STM)}

This mission addresses one of the most compelling questions in planetary science and astrobiology: \textit{What are the long-term endogenic and exogenic controls on the presence of liquid water on terrestrial planets?} (Decadal Survey Q10.3b). Our team’s concept—a lightweight robotic rover targeting the mid-latitudes of Mars—is strategically designed to explore the interplay between Mars’s internal geologic processes, external climatic forces, and the current distribution and persistence of water ice. The mission supports both robotic and human exploration, fulfilling top-priority stakeholder needs outlined in NASA’s Mission Task Document and Decadal Strategy.

The science traceability process began with a detailed examination of the Q10.3b prompt, followed by the derivation of two mission-specific science objectives. The first objective focuses on the detection and geophysical characterization of subsurface water ice, while the second targets understanding how geological and climatic history influence the ice’s stability, location, and potential habitability. These objectives are aligned with the strategic needs for Mars exploration, particularly in evaluating possible human landing sites and long-term outpost support.

The traceability matrix baselines the first three columns—science goals, objectives, and measurements—ensuring a clear, logical flow from high-level priorities to mission implementation. Subsequent columns include physical parameters, observables, performance requirements, instruments, and mission constraints. All measurements are designed to be feasible within the constraints of a Discovery-class rover profile (≤200 kg, ≤2.5 m\textsuperscript{3}, ≤\$450M).

\subsection*{Science Objective 1: Detect Subsurface Water Ice}

The primary method for identifying subsurface ice is radar sounding, executed via a forward-facing ground-penetrating radar modeled after SHARAD and RIMFAX. This instrument will detect dielectric interfaces corresponding to water ice layers up to 10 meters deep. Complementary to this, a neutron spectrometer—derived from DAN or NS—will passively detect hydrogen signatures as indirect indicators of water presence. Together, these instruments provide volumetric and compositional data at decimeter-scale resolution.

To provide further context, surface spectrometers (VNIR or TIR) will identify hydrated minerals such as perchlorates and sulfates, offering insights into historical water-rock interactions. In Moreux Crater, glacial features and lobate debris aprons can be linked to mineralogical evidence of past ice movement and melting.

\subsection*{Science Objective 2: Understand Controls on Water Distribution}

Identifying water is only the first step—understanding its distribution and stability is critical. This objective requires integrating geological and climatic data. The rover will assess regolith thermal properties, atmospheric pressure, and diurnal temperature cycles to determine whether detected ice is currently stable or a remnant of past epochs. Subsurface temperature gradients will be inferred via thermal inertia modeling and validated with onboard environmental sensors.

Topographic and geomorphological mapping with stereo cameras and radar will reveal how crater slopes, elevation, and insulating regolith thickness influence ice preservation. By correlating observed ice depths with geologic unit boundaries, the mission will identify controlling factors such as lithologic variation or glacial resurfacing.

These findings directly support ISRU planning, inform site safety for future human landings, and contribute to broader planetary evolution models. Whether subsurface ice is stable over decadal timescales or sensitive to obliquity-driven changes will critically affect the sustainability of surface infrastructure on Mars.

\subsection*{Stakeholder Satisfaction}

This STM supports both robotic science and exploration readiness. All objectives align with the Decadal Survey, MEPAG goals, and the Mission Task Document. The mission will produce high-value data to inform the selection of future crewed landing sites by resolving key questions about water accessibility and environmental stability. The STM is fully baselined, with TBDs limited to instrument performance parameters or mission requirement margins—each to be addressed in the TBD/TBR Resolution Table located in the appendix.