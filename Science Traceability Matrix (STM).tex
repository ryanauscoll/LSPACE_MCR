\section*{1.2 Science Traceability Matrix (STM)}
\addcontentsline{toc}{section}{1.2 Science Traceability Matrix (STM)}

This mission addresses one of the most compelling questions in planetary science and astrobiology: \textit{What are the long-term endogenic and exogenic controls on the presence of liquid water on terrestrial planets?} (Decadal Survey Q10.3b)\footnote{See \cite{nrc_2022_decadal}}. The team's concept, a lightweight robotic rover that targets mid-latitudes of Mars, is strategically designed to explore the interaction between Mars's internal geologic processes, external climatic forces, and the current distribution and persistence of water ice. The mission supports robotic and human exploration, meeting the highest priority stakeholder needs outlined in NASA’s Mission Task Document and Decadal Strategy\footnote{See \cite{lspace_stm_module, mepag_goals}}. \\

The science traceability process began with a detailed examination of the Q10.3b prompt, followed by the derivation of two mission-specific science exploration objectives, and two human exploration objectives. The first objective focuses on the detection and geophysical characterization of subsurface water ice, while the second targets understanding how geological and climatic history influence the ice’s stability, location, and potential habitability. These objectives are aligned with the strategic needs for Mars exploration, particularly in evaluating possible human landing sites and long-term outpost support\footnote{See \cite{nasa_m2m_objectives}}. \\

The traceability matrix baselines the first three columns, science goals, objectives, and measurements, ensuring a clear, logical flow from high-level priorities to mission implementation. Subsequent columns include physical parameters, observables, performance requirements, instruments, and mission constraints. All measurements are designed to be feasible within the constraints of a Discovery-class rover profile (≤200 kg, ≤2.5 m\textsuperscript{3}, ≤\$450M). \\

\subsection*{Science Objective 1: Detect Subsurface Water Ice}

The primary method for identifying subsurface ice is radar sounding, executed via a forward-facing ground-penetrating radar modeled after SHARAD and RIMFAX\footnote{See \cite{plaut_2007_subsurface, putzig_2012_stratigraphy}}. This instrument can detect dielectric interfaces corresponding to water ice layers up to 10 meters deep. Complementary to this, a neutron spectrometer—derived from DAN or NS can passively detect hydrogen signatures as indirect indicators of water presence\footnote{See \cite{nasa_rad}}. Together, these instruments could provide volumetric and compositional data at decimeter-scale resolution. \\

To provide further context, surface spectrometers (VNIR or TIR) can identify hydrated minerals such as perchlorates and sulfates, offering insights into historical water-rock interactions. In Moreux Crater, glacial features and lobate debris aprons can be linked to mineralogical evidence of past ice movement and melting\footnote{See \cite{bramson_2015_excess, mellon_1995_ground_ice}}.

\subsection*{Science Objective 2: Understand Controls on Water Distribution}

Identifying water is only the first step— understanding its distribution and stability is critical. This objective requires integrating geological and climatic data. The rover will assess regolith thermal properties, atmospheric pressure, and diurnal temperature cycles to determine whether detected ice is currently stable or a remnant of past epochs. Subsurface temperature gradients will be inferred via thermal inertia modeling and validated with onboard environmental sensors. \\

Topographic and geomorphological mapping with stereo cameras and radar will reveal how crater slopes, elevation, and insulating regolith thickness influence ice preservation. By correlating observed ice depths with geologic unit boundaries, the mission will identify controlling factors such as lithologic variation or glacial resurfacing\footnote{See \cite{head_2010_glaciation}}. \\

These findings directly support ISRU planning, inform site safety for future human landings, and contribute to broader planetary evolution models. Whether subsurface ice is stable over decadal timescales or sensitive to obliquity-driven changes will critically affect the sustainability of surface infrastructure on Mars.

\subsection*{Stakeholder Satisfaction}

This STM supports both robotic science and exploration readiness. All objectives align with the Decadal Survey, MEPAG goals, and the Mission Task\footnote{See \cite{nrc_2022_decadal, mepag_goals, lspace_stm_module}}. The mission will produce high-value data to inform the selection of future crewed landing sites by resolving key questions about water accessibility and environmental stability. The STM is fully baselined, with TBDs limited to instrument performance parameters or mission requirement margins—each to be addressed in the TBD/TBR Resolution Table.




\begin{table}[H]
\caption{Science Traceability Matrix (STM)}
\hspace{-1cm}
\scalebox{1.1}{
\renewcommand{\arraystretch}{1.5}
\scriptsize
\resizebox{\textwidth}{!}{%
\begin{tabular}{|C{4cm}|C{4cm}|C{4.5cm}|C{4cm}|C{3cm}|C{2.5cm}|C{2.5cm}|C{2.2cm}|C{2.2cm}|}
\hline
\textbf{Science Goals} & \textbf{Science Objectives} & \textbf{Physical Parameters} & \textbf{Observables} & \textbf{Science Measurement Requirements} & \textbf{Instrument Performance Requirements} & \textbf{Predicted Instrument Performance} & \textbf{Instrument} & \textbf{Mission Requirements} \\
\hline

\multirow{3}{=}{{HBS-1LM: Understand the effects of short- and long-duration exposure to the environments of the Moon, Mars, and deep space on biological systems and health, using humans, model organisms, systems of human physiology, and plants.}} 
& Determine the impact of Martian dust on human respiratory systems by characterizing the toxicity, particle size distribution, and chemical composition of airborne particles. 
& Identify if hexavalent Chromium is present in Martian soil and airborne dust at more than 150 ppm & Collect UV-visible absorbance spectra between 350–450 nm from soil and airborne dust samples processed every 6 solar hours. & TBD 1 & TBD 2 & TBD 5 & TBD 6 & TBD 9 \\
\cline{2-9}
& Assess the radiation environment and its effects on biological systems by measuring cosmic ray flux and cumulative dose rates at the Martian surface.
& Determine the cumulative radiation dose absorbed by phantom lung tissue over a 50-sol period on the Martian surface, including radiation contributions from GCR and solar particle events. & Quantify the energy deposition rate and flux of ionizing radiation in particles/cm²/s in phantom tissues across a 10 MeV-1 GeV range, with data collected hourly. & TBD 1 & TBD 2 & TBD 5 & TBD 6 & TBD 9 \\
\hline

\multirow{2}{=}{{Q10.3b: What are the long-term endogenic and exogenic controls on the presence of liquid water on terrestrial planets?}} 
& Determine if there is liquid water on or near the surface through geochemical measurements of ice and hydrous minerals and geophysical measurements of the crust.
& Identify the presence of water-bearing minerals, including Fe/Mg and Al-rich phyllosilicates and sulfates, in surface and near-surface regolith within a 100 m² study area.  & Measure reflectance spectra between 1.9–2.5 µm with mineral abundances resolved at a spatial resolution of 1 m. & TBD 3 & TBD 4 & TBD 7 & TBD 8 & TBD 10 \\
\cline{2-9}
& Determine the distribution, history, and processes driving the availability of ice and liquid water.
& Determine spatial variability in the deuterium-to-hydrogen (D/H) ratio of subsurface ice to assess contributions from distinct ancient water reservoirs and the regional history of water loss (Cockell, et all 2016). & Measure isotopic ratios of H and D in vapors released from subsurface ice at 5-meter lateral intervals, with a resolution of 0.1‰ & TBD 3 & TBD 4 & TBD 7 & TBD 8 & TBD 10 \\
\hline

\end{tabular}
}}
\end{table}


