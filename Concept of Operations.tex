\section*{1.7 Concept of Operations}
\addcontentsline{toc}{section}{1.7 Concept of Operations}

This Rover Mission aims to study the Martian surface with the objective of gathering data on the effects of exposure to the Martian environment on biological systems. Mars is known for its harsh conditions, including extreme temperatures, high radiation, low atmospheric pressure, lack of oxygen, and frequent dust storms. These conditions make long-term human habitation challenging\footnote{See \cite{nasa_mars_goals}}. This mission focuses on collecting samples and identifying sustainable water sources for future crewed missions\footnote{See \cite{nasa_m2m_objectives}}.\\

To achieve these goals, the team proposes a rover capable of landing within 20 feet of the designated target. The rover will be equipped to traverse the surface, detect shallow subsurface ice, assess environmental hazards, and test materials relevant to human spaceflight applications.\\

The mission timeline anticipates a travel duration of approximately 200 Earth days from launch to Martian surface arrival\footnote{See \cite{nasa_how_long_mars}}, leveraging industry contractor launch services and optimal planetary alignment windows. Upon touchdown, the rover will enter a boot-up phase, initiating system health checks to confirm survival of launch and landing stresses. Following successful verification, the rover will deploy its solar panels and instrumentation to begin charging and preparing for nominal surface operations.\\

Each Martian sol (day), the rover will perform the following operational phases: \textbf{Wake-Up, Health Checks, Scientific Operations, Shutdown, and Sleep Mode}. During Wake-Up, systems are powered on, batteries are charged, and communication links are established with Earth via a high-gain antenna. Health checks follow, ensuring operational integrity before scientific activities commence.\\

Scientific operations—estimated at six hours per sol—include surface sampling, spectrometry, subsurface probing, and environmental monitoring. One major objective is to search for evidence of a sustained water source. This includes identifying carbonate minerals, which form in the presence of water and carbon dioxide. Their detection could indicate that Mars once harbored stable liquid water, potentially supporting microbial life\footnote{See \cite{nasa_mars_goals}}.\\

After scientific operations conclude, the rover enters a controlled Shutdown phase. Systems are powered down to reduce energy consumption, and the rover transitions into Sleep Mode overnight. This cycle repeats daily for the duration of the mission\footnote{See \cite{nasa_mer_timeline}}.\\

The mission lifespan is expected to be up to 10 Earth years, though actual longevity will depend on the degradation rate of the solar arrays, which are susceptible to Martian dust accumulation. Mission conclusion will occur once all science objectives are met or when the rover can no longer generate sufficient power for operations. All data will be transmitted back to Earth, supporting planning and safety analysis for future crewed missions, including optimal site selection for human habitats\footnote{See \cite{nasa_m2m_objectives}}.