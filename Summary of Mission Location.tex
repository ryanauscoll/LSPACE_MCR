
\section*{1.3 Summary of Mission Location}
\addcontentsline{toc}{section}{1.3 Summary of Mission Location}

Mars, the fourth planet from the Sun, is a cold, terrestrial world with a thin carbon dioxide (CO\textsubscript{2}) atmosphere. Its surface is shaped by volcanic activity, impact cratering, aeolian erosion, and seasonal freeze-thaw processes. The planet experiences extreme temperature variations, frequent dust storms, and high radiation levels due to the absence of a global magnetic field and protective upper atmosphere. Despite these challenges, Mars remains a prime candidate for exploration, largely because of its potential to harbor accessible water ice in the near subsurface.

Orbital datasets from missions such as Mars Odyssey, the Mars Reconnaissance Orbiter (MRO), and the Subsurface Water Ice Mapping (SWIM) project have revealed compelling evidence of near-surface hydrogen-rich regolith in the northern mid-latitudes. These regions are believed to contain water ice at depths of less than one meter—ideal for both scientific investigation and in situ resource utilization (ISRU) in support of future human missions. The latitude range also offers favorable thermal conditions and sufficient solar availability for rover operations.

This mission targets a 10 km × 10 km Exploration Zone (EZ) located within one of NASA’s high-priority shallow ice candidate zones, as defined by the SWIM maps. The region will be selected based on three key criteria:

\begin{itemize}
    \item Presence of subsurface water ice at 0–1 m depth
    \item Geologic features indicative of volatile exposure or preservation
    \item Navigable terrain within rover mobility and risk constraints
\end{itemize}

Candidate features within the EZ may include:

\begin{itemize}
    \item \textbf{Recent impact craters} — may expose subsurface volatiles via fresh ejecta
    \item \textbf{Polygonal terrain} — indicates thermal contraction and seasonal freeze-thaw cycles
    \item \textbf{Gently sloped surfaces} — improve rover stability and may help preserve underlying ice
    \item \textbf{Spectral signatures from CRISM and SHARAD} — suggest hydrated minerals or water-bearing deposits
\end{itemize}

Site selection will be finalized using NASA’s JMARS (Java Mission-planning and Analysis for Remote Sensing) tool, informed by overlays from SWIM, CRISM mineralogical maps, elevation models, and slope data. Final considerations will include engineering feasibility, astrobiological interest, planetary protection constraints, and accessibility for surface mobility.

Targeting this terrain directly advances NASA’s Moon to Mars objectives and the Mars Exploration Program Analysis Group (MEPAG) science goals by identifying accessible water resources for ISRU, while also informing future long-duration surface operations and life-detection strategies.
